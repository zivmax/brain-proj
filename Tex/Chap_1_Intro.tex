\chapter{引言}\label{chap:introduction}

\section{脑类器官的理解}\label{sec:brain-organoid}
脑类器官是一种在体外人工培养的组织,其形态和功能类似于人类大脑的某些部分\cite{Kim2023}。通过干细胞技术,科学家能够在实验室中培养出这些微型脑组织,用于研究大脑发育、疾病机制以及药物筛选等领域。脑类器官的出现为神经科学研究提供了一个高度可控的实验平台,尤其是在研究复杂神经系统疾病时,其优势尤为明显。

\section{研究目的}\label{sec:research-purpose}
脑类器官的主要目的是在更加可控的环境中研究与大脑相关的疾病。传统的动物模型虽然在一定程度上能够模拟人类大脑的功能,但由于物种差异,其研究结果往往难以直接应用于人类。而脑类器官则能够更好地模拟人类大脑的发育和功能,为研究神经系统疾病提供了新的可能性。

\section{面临的挑战}\label{sec:research-challenges}
尽管脑类器官在研究中展现出巨大的潜力,但其发展仍面临诸多挑战。首先,从物理结构上来看,脑类器官的构建非常复杂。大脑是一个高度复杂的器官,包含多个功能区域和复杂的神经网络,如何在体外精确重建这些结构是一个巨大的难题。其次,从功能角度来看,脑类器官的功能发育也面临挑战。即使能够构建出形态上相似的脑类器官,如何使其具备与真实大脑相似的功能仍然是一个未解之谜。

\section{可能的解决方案}\label{sec:research-solutions}
针对上述挑战,研究者提出了两种可能的解决方案。首先,从物理结构的角度,可以通过专注于重建特定的大脑区域来简化问题。例如,选择重建海马体或前额叶皮层等特定区域,而不是试图重建整个大脑。其次,从功能发育的角度,可以通过主动“训练”脑类器官,使其接受类似于真实大脑的完整刺激,从而促进其功能发育。这种训练可以通过电刺激、化学刺激或光遗传学等手段实现。

