\chapter{一种创新研究方法}\label{chap:methodlogy}
在本研究中,我们提出了一种创新的重建方法,基于前人的研究进一步尝试在微观的神经元连接层面进行重建,以得到完整的功能性脑类器官。

\section{宏观形态的重建}\label{sec:macro-reconstruction}
通过干细胞技术,我们探索将人类诱导多能干细胞(hiPSC)分化为特定脑区神经元的新途径,并在体外培养出具有特定形态和对应标志物的脑类器官。

\section{微观连接的重建}\label{sec:micro-reconstruction}
通过引入外部刺激,如电刺激和化学刺激,设计了一种训练机制,促进脑类器官中神经元之间的连接形成。用来模拟真实大脑的学习过程,使得脑类器官能够形成功能性的神经网络。

\section{功能性验证}\label{sec:functional-verification}
参考“盘中之脑”的研究,我们通过记录脑类器官的电生理信号,作为输入输出的信号载体,结合自由能原理,验证脑类器官的功能性。