%---------------------------------------------------------------------------%
%->> 封面信息及生成
%---------------------------------------------------------------------------%
%-
%-> 中文封面信息
%-
\confidential{}% 密级:只有涉密论文才填写
\schoollogo{scale=0.2}{ShanghaiTechLogo}% 校徽
%======论文题目 页眉显示 务必准确========
\title{功能性脑类器官的工程化构建}% 论文中文题目
\author{唐林峥、虞果、刘思昀}% 论文作者
\ID{2022533087、2022533174、2022522011}
\entranceYear{2022}
\advisor{张寒}% 指导教师:姓名
\advisorsec{}% 第二指导老师:按情况填写
\degree{学士}% 学位:学士、硕士、博士
\degreetype{理学}% 学位类别:理学、工学、工程、医学等
\major{脑科学与脑疾病}% 二级学科专业名称
\institute{生物医学工程学院}% 院系名称
\chinesedate{二零二五年~一月}% 毕业日期:夏季为6月、冬季为12月
%-
%-> 英文封面信息
%-
\englishtitle{Engineering Functional Brain Organoids}% 论文英文题目
\englishauthor{Lin-Zheng Tang, Guo Yu, Si-Yun Liu}% 论文作者
\englishadvisor{Han Zhang}% 指导教师
\englishdegree{Bachelor}% 学位:Bachelor, Master, Doctor。封面格式将根据英文学位名称自动切换,请确保拼写准确无误
\englishdegreetype{Natural Science}% 学位类别:Philosophy, Natural Science, Engineering, Economics, Agriculture 等
\englishthesistype{thesis}% 论文类型: thesis, dissertation
\englishmajor{Brain Science and Brain Disorders}% 二级学科专业名称
\englishinstitute{School of Biomedical Engineering}% 院系名称
\englishdate{January 2025}% 毕业日期:夏季为June、冬季为December
%-
%-> 生成封面
%-
%-======================================================
% 封面和声明页格式不符合教务处要求,因此此处不生成pdf格式的 ||
% 论文封面和声明页,请各位使用word模板中的相关内容。       ||
%=======================================================
\maketitle% 生成中文封面
\makeenglishtitle% 生成英文封面
%-

\chapter*{摘\quad 要}\chaptermark{摘\quad 要}% 摘要标题
\setcounter{page}{1}% 开始页码
\pagenumbering{Roman}% 页码符号

本文探讨了基于自由能原理(Free Energy Principle, FEP)的智能脑类器官开发方法,旨在通过体外培养功能性神经网络来模拟人类大脑的发育和功能。脑类器官作为一种人工培养的组织,能够在实验室环境中模拟特定脑区的结构和功能,为研究神经系统疾病提供了新的实验平台。本文首先介绍了脑类器官的基本概念及其在神经科学研究中的重要性,随后详细阐述了自由能原理的理论框架及其在脑类器官训练中的应用。通过结合电刺激和化学刺激,本文提出了一种创新的脑类器官训练方法,并设计了一个基于海马体脑类器官的抑郁症模型。该模型不仅为抑郁症的机制研究提供了新的实验平台,还可用于药物筛选和网络水平变化的研究。

\keywords{脑类器官,自由能原理,神经网络,抑郁症模型,药物筛选}% 中文关键词

%-
%-> 英文摘要
%-
\chapter*{Abstract}\chaptermark{Abstract}% 摘要标题

This paper explores the development of intelligent brain organoids based on the Free Energy Principle (FEP), aiming to simulate the development and functionality of the human brain by cultivating functional neural networks in vitro. Brain organoids, as artificially grown tissues, can mimic the structure and function of specific brain regions in a laboratory setting, providing a novel experimental platform for studying neurological disorders. The paper begins by introducing the basic concepts of brain organoids and their significance in neuroscience research. It then elaborates on the theoretical framework of the Free Energy Principle and its application in training brain organoids. By combining electrical and chemical stimuli, this paper proposes an innovative method for training brain organoids and designs a depression model based on hippocampal organoids. This model not only offers a new experimental platform for studying the mechanisms of depression but also facilitates drug screening and research on network-level changes.

\englishkeywords{Brain Organoids, Free Energy Principle, Neural Networks, Depression Model, Drug Screening}% 英文关键词
%---------------------------------------------------------------------------%
